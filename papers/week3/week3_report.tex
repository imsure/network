%%%%%%%%%%%%%%%%%%%%%%%%%%%%%%%%%%%%
% LaTeX template for reading report
% Author: Shuo Yang
%%%%%%%%%%%%%%%%%%%%%%%%%%%%%%%%%%%%

\documentclass[11pt]{article}
\usepackage{amsmath,amssymb,epsfig,graphics,hyperref,amsthm,mathtools,enumitem}
\DeclarePairedDelimiter\ceil{\lceil}{\rceil}
\DeclarePairedDelimiter\floor{\lfloor}{\rfloor}

\hypersetup{colorlinks=true}

\setlength{\textwidth}{7in}
\setlength{\topmargin}{-0.575in}
\setlength{\textheight}{9.25in}
\setlength{\oddsidemargin}{-.25in}
\setlength{\evensidemargin}{-.25in}

\reversemarginpar
\setlength{\marginparsep}{-15mm}

\newcommand{\rmv}[1]{}
\newcommand{\bemph}[1]{{\bfseries\itshape#1}}
\newcommand{\N}{\mathbb{N}}
\newcommand{\Z}{\mathbb{Z}}
\newcommand{\imply}{\to}
\newcommand{\bic}{\leftrightarrow}

% Some user defined strings for the homework assignment
%
\def\CourseCode{CS525}
\def\ReportNo{3}
\def\Category{Reading Report}
\def\PaperTitle{The Revised ARPANET Routing Metric}
\def\Author{Shuo Yang}

\begin{document}

\noindent

\CourseCode \hfill \Category

\begin{center}
Reading Report \#\ReportNo\\
Paper: \PaperTitle\\
Student: \Author\\
\end{center}

% A horizontal split line
\hrule\smallskip
\vspace{1.5em}

The revised metrics and its implementation HN-SPF proposed by the
paper improved routing in the ARPANET substantially compared to the
previous alternatives (min-hop and D-SPF). However, the paper does not
give good enough technical details for others to repeat or improve their
work. It also leaves some open questions yet to be answered. Therefore
I will first list weaknesses of the paper and then state what I would
do to make up these weaknesses.

First, the paper is being vague about how they chose
parameters/coefficients for the revised metric.
For example, it limits the relative cost for a link can
report to be no more than two additional hops in a homogeneous
network. No justification was made for this number.
In addition, the revised metric puts upper and
lower bounds on the cost that can be reported by each link. But they
did not talk about how they chose this upper and lower bound, what
criteria they used and how they would justify their choices. Further,
how to adjust normalization parameters based on link type remains
unanswered. Is it done by 
hand or by program? Is it static for each link type or dynamically
adaptive?
Without knowing these, it is very hard to repeat their work. Moreover,
we need answers to these questions in order to evaluate if the revised
metrics can be scaled to more diverse networks with many link
types. For complex networks, hand-tuning parameters sounds a horrible
idea. 


Second, the pseudo code for HN-SPF is hard to understand without being
given more detailed explanations.

Third, the paper shows that HN-SPF can oscillate around its
equilibrium with a bounded amplitude. But it does not talk about how
we could eliminate this problem. This minor oscillation can cause more
network traffic.

To address the first problem, I will elaborate the reasons for
choosing those normalization parameters. Try to argue that they are
optimal or sub-optimal, or give theoretical background to justify the
choices, or at least give heuristics for hand-tuning those
numbers to achieve better performance.

To address the second problem, I will rewrite the pseudo code for
HN-SPN to make it easier for readers to understand and give more
detailed explanations about it. Thus they can reproduce the
work for further improvement.

To address the third problem, I will talk about how we could possibly
eliminate the oscillation around the equilibrium. My thinking is that
if each router is aware of as soon as it enters the oscillation region,
we can avoid oscillation by using some different metrics calculation
strategy or by changing the way network responses. For example, we can
employ a different link cost function which only applies to the
oscillation region. It can be as simple as changing the update limits
(maximum \& minimum changes allowed) for link cost value, or can be a
complicated formula. The idea is that the routing metrics should
be adaptive to the dynamic changes in networks. The assumption,
though, is that routers are sensitive to the equilibrium state and
oscillation region.

\end{document}
