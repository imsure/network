%%%%%%%%%%%%%%%%%%%%%%%%%%%%%%%%%%%%
% LaTeX template for reading report
% Author: Shuo Yang
%%%%%%%%%%%%%%%%%%%%%%%%%%%%%%%%%%%%

\documentclass[11pt]{article}
\usepackage{amsmath,amssymb,epsfig,graphics,hyperref,amsthm,mathtools,enumitem}
\DeclarePairedDelimiter\ceil{\lceil}{\rceil}
\DeclarePairedDelimiter\floor{\lfloor}{\rfloor}

\hypersetup{colorlinks=true}

\setlength{\textwidth}{7in}
\setlength{\topmargin}{-0.575in}
\setlength{\textheight}{9.25in}
\setlength{\oddsidemargin}{-.25in}
\setlength{\evensidemargin}{-.25in}

\reversemarginpar
\setlength{\marginparsep}{-15mm}

\newcommand{\rmv}[1]{}
\newcommand{\bemph}[1]{{\bfseries\itshape#1}}
\newcommand{\N}{\mathbb{N}}
\newcommand{\Z}{\mathbb{Z}}
\newcommand{\imply}{\to}
\newcommand{\bic}{\leftrightarrow}

% Some user defined strings for the homework assignment
%
\def\CourseCode{CS525}
\def\ReportNo{1}
\def\Category{Reading Report}
\def\PaperTitle{On Distributed Communications Networks}
\def\Author{Shuo Yang}

\begin{document}

\noindent

\CourseCode \hfill \Category

\begin{center}
Reading Report \#\ReportNo\\
Paper: \PaperTitle\\
Student: \Author\\
\end{center}

% A horizontal split line
\hrule\smallskip
\vspace{1.5em}
I would like to \textbf{criticize} that the author did not give a lot of weight
to the \textbf{scalability} of the networks in his paper. In the 
\emph{Introduction} section, he brought up three types of networks:
\emph{Centralized}, \emph{Decentralized} and \emph{Distributed
  networks}. After briefly describing the drawbacks of centralized and
decentralized networks in terms of survivability after nuclear
attack, he completely turned into the examination of the distributed
networks. It is very understandable that under that specific cold war
context, survivability should be the main focus.  
However, it is not convincing to me that decentralized
networks are also vulnerable, the reason ``destruction of a small
number of nodes in a decentralized network can destroy
communications'' is not sound enough. It really depends on the scale
of the networks. Actually, the decentralized
topology looks much like the current Internet. And at the time the
paper was written, building a complete distributed network might not be
feasible.

I am interested in this question: ``\emph{What was the scale of the network
the author intended to build in reality?}''. The author seemed being
vague about this. He just mentioned a $18\times18$-array network was
built for simulation, which is definitely not a large-scale network.

I believe that scalability is important for choosing the right
structure for networks. Building a large-scale complete distributed
network with redundancy level of 3 or 4 might not be feasible. On the
other hand, decentralized topology seems appealing to me in the way
that it manages a group of computers (all connected to a local switch)
in a cloud and let the switch talk to the outside. In doing this, we
can build large-scale networks recursively. This is basically how we
build Internet today.

Since at the end of the paper, the author briefly envisioned the
future complex networks that could span continents,  
I would like to do the following to follow up his work:\\

\begin{enumerate}[nolistsep]
\item Talk about the importance of the scalability for computer
  networks.
\item Simulate a decentralized network with a relatively large size. I
  will make the redundancy level very high (5 or 6) between switch
  nodes. Then simulate a distributed network with the same size with
  redundancy level of 3 or 4.
\item Evaluate and compare the two types of networks based on the
  simulation results.
\item Argue that though for small scale networks,
  decentralized topology is not as reliable as the distributed one, for
  relatively large scale networks, decentralized topology is good
  enough to survive after destruction because there are many switch
  nodes in the network and redundancy level among them is very high.
\item Argue that decentralized networks are much more scalable than
  complete distributed networks, so it is much more feasible to build
  large-scale decentralized networks in reality.
\item Argue that the separation of computer nodes and switch nodes
  makes it easier to provide common services to network users and
  build logical channels between end users. While in a distributed
  networks, every node basically is a switch, it needs to take
  care of message routing itself. The separation of concerns in the
  decentralized networks provides a better abstraction.
\end{enumerate}

\end{document}
