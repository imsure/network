%%%%%%%%%%%%%%%%%%%%%%%%%%%%%%%%%%%%%%
% LaTeX template for summary of papers
% Author: Shuo Yang
%%%%%%%%%%%%%%%%%%%%%%%%%%%%%%%%%%%%%%

\documentclass[11pt]{article}
\usepackage{amsmath,amssymb,epsfig,graphics,hyperref,amsthm,mathtools,enumitem}
\DeclarePairedDelimiter\ceil{\lceil}{\rceil}
\DeclarePairedDelimiter\floor{\lfloor}{\rfloor}

\hypersetup{colorlinks=true}

\setlength{\textwidth}{7in}
\setlength{\topmargin}{-0.575in}
\setlength{\textheight}{9.25in}
\setlength{\oddsidemargin}{-.25in}
\setlength{\evensidemargin}{-.25in}

\reversemarginpar
\setlength{\marginparsep}{-15mm}

\newcommand{\rmv}[1]{}
\newcommand{\bemph}[1]{{\bfseries\itshape#1}}
\newcommand{\N}{\mathbb{N}}
\newcommand{\Z}{\mathbb{Z}}
\newcommand{\imply}{\to}
\newcommand{\bic}{\leftrightarrow}

% Some user defined strings for the homework assignment
%
\def\CourseCode{CS525}
\def\WeekNo{2}
\def\PaperNo{2}
\def\Category{Paper Summary}
\def\PaperTitle{End-to-End Argument in System Design}
\def\Author{Shuo Yang}

\begin{document}

\noindent

\CourseCode \hfill \Category

\begin{center}
Week \#\WeekNo\\
Paper Summary \#\PaperNo\\
Paper: \PaperTitle\\
Student: \Author\\
\end{center}

% A horizontal split line
\hrule\smallskip

% First Pass:
% Carefully read abstract, intro, main design, and conclusions.
% Skim analysis/evaluation section.
\section{First Pass}

\emph{Carefully read abstract, intro, main design, and
  conclusions. Skim analysis/evaluation section.}

\subsection{What is the goal?}
The paper proposes a design principle called \emph{end-to-end
  argument} for distributed computer systems, articulates the argument
explicitly, so as to examine its nature and to see how general it
really is. 

\subsection{What is the specific problem?}
Choosing the proper boundaries between functions is the key for
computer system design. 

\subsection{What is the context (background, previous work,
  assumption)?} 
\emph{Background}: a class of function placement argument has been
used for many years with neither explicit recognition nor much
conviction;the emergence of the computer networks.\\
\emph{Assumptions}: The function in question can completely and
correctly be implemented only with the knowledge and help of the
application standing at the end points of the communication system.

\subsection{What is the main idea of the paper?}
The argument appeals to application requirements, and provides a
rationale for moving function upward in a layered system, closer to
the application that uses the function. The lower levels need not
provide ``perfect'' reliability.


% Second Pass:
% Start with the assumption that the main point is wrong, even if you
% agree with it.
% Focus on analysis/evaluation sections and the details of the design.
% How do the results prove otherwise?
% What are the weak points?
\section{Second Pass}

\emph{Start with the assumption that the main point is wrong, even if
  you agree with it. Focus on analysis/evaluation sections and the
  details of the design. (How do the results prove otherwise?) What
  are the weak points?}

\subsection{How do the results prove otherwise?}
\emph{How do the results prove that my assumption (that the main point
  is wrong) is wrong?}

Start with an example of ``reliable file transfer'', the authors argues
even if the communication system provides low-level reliable data
transmission, it still cannot guarantee the reliability since correct
file transmission is assured by the end-to-end checksum and retry
whether or not the data transmission system is especially reliable.

Based on this example, the authors argue that ``in order to achieve
careful file transfer, the application program that performs the
transfer must supply a file-transfer-specific, end-to-end reliability
guarantee''.

Then from performance aspects, the authors argue that the lower levels
need not provide ``perfect'' reliability. Thus the amount of effort to
put into reliability measures within the data communication system is
seen to be an engineering tradeoff based on performance, rather than a
requirement for correctness. But authors raise the concern of
justifying the placing of functions in low-level subsystem using
performance, that is: 1) the lower level subsystem is
common to many applications, those applications that do not need the
function will pay for it anyway; 2) the low-level subsystem may not
have as much information as the higher levels, so it cannot do the job
as efficiently.

Then the authors use serveral other examples to justify their point.

\subsection{What are the weak points?}
\begin{enumerate}
\item Most arguments are based on file transfer example.
\item Did not draw the scope of network applications.
\item What are application functions?
\item ?
\end{enumerate}

\end{document}
